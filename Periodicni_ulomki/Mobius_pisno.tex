\documentclass[a4paper,12pt]{article}

\usepackage[slovene]{babel}
\usepackage{amsfonts,amssymb,amsmath}
\usepackage[utf8]{inputenc}
\usepackage[T1]{fontenc}
\usepackage{lmodern}
\usepackage{graphicx}


\def\N{\mathbb{N}} % mnozica naravnih stevil
\def\Z{\mathbb{Z}} % mnozica celih stevil
\def\Q{\mathbb{Q}} % mnozica racionalnih stevil
\def\R{\mathbb{R}} % mnozica realnih stevil
\def\C{\mathbb{C}} % mnozica kompleksnih stevil
\newcommand{\geslo}[2]{\noindent\textbf{#1} \quad \hangindent=1cm #2\\[-1pc]}

\def\qed{$\hfill\Box$}   % konec dokaza
\def\qedm{\qquad\Box}   % konec dokaza v matematičnem načinu
\newtheorem{izrek}{Izrek}
\newtheorem{trditev}{Trditev}
\newtheorem{posledica}{Posledica}
\newtheorem{lema}{Lema}
\newtheorem{pripomba}{Pripomba}
\newtheorem{definicija}{Definicija}
\newtheorem{zgled}{Zgled}

\title{M\"{o}biusove transformacije in periodični verižni ulomki \\ 
\Large Seminar}
\author{Nejc Zajc \\
Fakulteta za matematiko in fiziko \\
Oddelek za matematiko}
\date{17.\ marec 2020}

\begin{document}


%%%%%%%%%%%%%%%%%%%%%%%%%%%%%%%%%%%%%%%%%%%%%%%%%%%%%%%%%%%%%%%%%%%%%


\maketitle


%%%%%%%%%%%%%%%%%%%%%%%%%%%%%%%%%%%%%%%%%%%%%%%%%%%%%%%%%%%%%%%%%%%%%

\section{Uvod}

Verižne ulomke pogosto spoznavamo pri teoriji števil, kjer so členi naravna števila. Ob ukvarjanju s števili oblike
\[
    b_0 + \cfrac{a_1}{b_1 + \cfrac{a_2}{b_2 + \ddots}},
\]
kjer so $a_i$ in $b_i$ kompleksna števila, hitro opazimo potrebo po novih pristopih. 


\begin{definicija}
    M\"{o}biusova transformacija je kompleksna funkcija oblike 
    \begin{equation}
    \label{Mob_def}
        g(z) = \frac{az + b}{cz + d},
    \end{equation}
    kjer so $a$, $b$, $c$ in $d$ kompleksna števila, za katera velja $ad - bc \neq 0$.
\end{definicija}

Pomen teh funkcij opazimo, če si definiramo $s_1(z) = (az + 1)/z = a + 1/z$ in $s_2(z) = (bz + 1)/z = b + 1/z$; tedaj je namreč
\[
    s(z) = s_1(s_2(z)) = a + \cfrac{1}{b + \cfrac{1}{z}}  
\]
končen verižni ulomek in hkrati M\"{o}biusova transformacija. Transformacije bomo uporabili za dokaz glavnega izreka, za začetek pa si natančneje oglejmo verižne ulomke.

%%%%%%%%%%%%%%%%%%%%%%%%%%%%%%%%%%%%%%%%%%%%%%%%%%%%%%%%%%%%%%%%%%%%%%%%%%%%%%%%%

\section{Verižni ulomki}

V članku se bomo ukvarjali z enostavnimi verižni ulomki. Enostaven verižni ulomek ima vse $a_i = 1$, zanj vpeljemo tudi krajši zapis, ki je v končni obliki enak
\[
    [b_0, b_1, \ldots, b_n] = b_0 + \cfrac{1}{b_1 + \cfrac{1}{\cdots + \cfrac{1}{b_n}}},
\]

kjer je $b_0$ celo število, $b_1, b_2, \ldots$ pa naravna števila; enostaven verižni ulomek pa je nato enak

\begin{equation}
\label{Ver_def}
    [b_0, b_1, b_2, \ldots] = b_0 + \cfrac{1}{b_1 + \cfrac{1}{b_2 + \ddots}} = \lim_{n \to \infty} [b_0, b_1, \ldots, b_n].
\end{equation}

Limita v \eqref{Ver_def} vedno obstaja, saj njegovi približki $[b_0, b_1, \ldots, c_n]$ strogo naraščajo za sode $n$ in so navzgor omejeni s $[b_0, b_1]$ ter posledično konvergirajo. Z lih $n$ ti približki strogo padajo in so omejeni z $[b_0]$ ter tako tudi konvergirajo. Ker je razlika med zaporednima približkoma obratno sorazmerna z $n$, sta limiti enaki, posledično pa limita \eqref{Ver_def} obstaja. 

Za verižni ulomek $[b_0, b_1, b_2, \ldots]$ rečemo, da je \textit{periodičen} s periodo $k$, če je $b_n = b_{n+k}$ za vsa naravna števila $n$, in da je \textit{ščasoma periodičen}, če je $b_n = b_{n+k}$ za vse dovolj velike $n$.




%%%%%%%%%%%%%%%%%%%%%%%%%%%%%%%%%%%%%%%%%%%%%%%%%%%%%%%%%%%%%%%%%%%%%


\section{Eulerjeva funkcija}


\begin{definicija}
aaabbbbbb
\end{definicija}

\begin{zgled}
bbb
\begin{table}[h]
\[
\begin{array}{clc}
 n & \{1,2,\ldots, n\}          & \varphi(n)       \\
 \hline
 1 & \{{\bf 1}\}                    &     1      \\
 2 & \{{\bf 1},2 \}                &     1      \\
 3 & \{{\bf 1,2},3 \}             &     2      \\
 4 & \{{\bf 1},2,{\bf 3},4 \} &     2      \\
 5 & \{{\bf 1,2,3,4},5 \}       &     4      \\
 6 & \{{\bf 1},2,3,4,{\bf 5},6 \} &     2
\end{array}
\] 
\caption{Vrednosti funkcije $\varphi(n)$ za $n = 1,2,\ldots,6$}\label{fi}
\end{table}


\end{zgled}



\begin{trditev}
\label{fipp}
fipp
\end{trditev}

\noindent
{\em Dokaz:\/} ccc
\qed


\begin{zgled}
aaa
\end{zgled}


\begin{eqnarray*}
\varphi(n) &=& \prod_{i=1}^r \varphi\left(p_i^{k_i}\right)
\ =\ \prod_{i=1}^r \left(p_i^{k_i} - p_i^{k_i-1}\right) \\
 &=& \left(\prod_{i=1}^r p_i^{k_i}\right) \times \prod_{i=1}^r \left(1 - \frac{1}{p_i}\right)
\ =\ n \times \prod_{p\,|\,n} \left(1 - \frac{1}{p}\right). \qedm
\end{eqnarray*}


\begin{izrek}[Eulerjev izrek]
Euler
\end{izrek}

\noindent
{\em Dokaz:\/} ddd \qed


%%%%%%%%%%%%%%%%%%%%%%%%%%%%%%%%%%%%%%%%%%%%%%%%%%%%%%%%%%%%%%%%%%%%%


\begin{eqnarray*}
(f * (g + h))(n) &=& \sum_{d e = n} f(d)(g+h)(e)
\ =\ \sum_{d e = n} f(d)(g(e)+h(e)) \\
 &=& \sum_{d e = n} f(d)g(e) + \sum_{d e = n} f(d)h(e) \\
 &=& (f * g + f * h)(n). \qedm
\end{eqnarray*}


\section*{Angleško-slovenski slovar strokovnih izrazov}


\geslo{arithmetic function}{aritmetična funkcija}

\geslo{coprime}{tuj}

\geslo{Dirichlet convolution}{Dirichletova konvolucija}

\geslo{Dirichlet ring}{Dirichletov kolobar, kolobar aritmetičnih funkcij}

\geslo{divisor}{delitelj}

\geslo{Euler's phi function, Euler's totient function}{Eulerjeva funkcija $\varphi$}

\geslo{Euler's theorem}{Eulerjev izrek}

\geslo{Fermat's little theorem}{mali Fermatov izrek}

\geslo{fundamental theorem of arithmetic}{osnovni izrek aritmetike}

\geslo{greatest common divisor}{največji skupni delitelj, največja skupna mera}

\geslo{least common multiple}{najmanjši skupni večkratnik}

\geslo{M\"obius function}{M\"obiusova funkcija $\mu$}

\geslo{M\"obius inversion}{M\"obiusov obrat, M\"obiusova inverzija}

\geslo{multiple}{večkratnik}

\geslo{prime}{praštevilo; praštevilski}

\geslo{prime factor}{prafaktor}

\geslo{prime number}{praštevilo}

\geslo{relatively prime}{tuj}




\begin{thebibliography}{1}
\bibitem{AiZ}
M.~Aigner in G.~M.~Ziegler, \emph{Proofs from THE BOOK}, 2.\ izdaja, Springer, Berlin--Heidelberg--New York, 2001.
\bibitem{CaW}
N.~Calkin in H.~S.~Wilf, Recounting the rationals,
\emph{Amer.~Math.~Monthly}  \textbf{107}  (2000),  360--363.
\bibitem{Gra}
J.~Grasselli, \emph{Elementarna teorija števil}, DMFA -- založništvo, Ljubljana, 2009.
\end{thebibliography}





\end{document}