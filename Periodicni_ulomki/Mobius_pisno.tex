\documentclass[a4paper,12pt]{article}

\usepackage[slovene]{babel}
\usepackage{amsfonts,amssymb,amsmath}
\usepackage[utf8]{inputenc}
\usepackage[T1]{fontenc}
\usepackage{lmodern}
\usepackage{graphicx}


\def\N{\mathbb{N}} % mnozica naravnih stevil
\def\Z{\mathbb{Z}} % mnozica celih stevil
\def\Q{\mathbb{Q}} % mnozica racionalnih stevil
\def\R{\mathbb{R}} % mnozica realnih stevil
\def\C{\mathbb{C}} % mnozica kompleksnih stevil
\newcommand{\geslo}[2]{\noindent\textbf{#1} \quad \hangindent=1cm #2\\[-1pc]}

\def\qed{$\hfill\Box$}   % konec dokaza
\def\qedm{\qquad\Box}   % konec dokaza v matematičnem načinu
\newtheorem{izrek}{Izrek}
\newtheorem{trditev}{Trditev}
\newtheorem{posledica}{Posledica}
\newtheorem{lema}{Lema}
\newtheorem{pripomba}{Pripomba}
\newtheorem{definicija}{Definicija}
\newtheorem{zgled}{Zgled}

\title{M\"{o}biusove transformacije in periodični verižni ulomki \\ 
\Large Seminar}
\author{Nejc Zajc \\
Fakulteta za matematiko in fiziko \\
Oddelek za matematiko}
\date{17.\ marec 2020}

\begin{document}


%%%%%%%%%%%%%%%%%%%%%%%%%%%%%%%%%%%%%%%%%%%%%%%%%%%%%%%%%%%%%%%%%%%%%


\maketitle


%%%%%%%%%%%%%%%%%%%%%%%%%%%%%%%%%%%%%%%%%%%%%%%%%%%%%%%%%%%%%%%%%%%%%

\section{Uvod}

Z verižnimi ulomki oblike
\[
    b_0 + \cfrac{a_1}{b_1 + \cfrac{a_2}{b_2 + \ddots}},
\]
se večinoma prvič srečamo pri teoriji števil, kjer so njihovi členi naravna števila. A ko to ne velja več in so členi poljubna kompleksna števila hitro opazimo potrebo po novih pristopih. Z verižnimi ulomki se lahko ukvarjamo s pomočjo M\"{o}biusovih transformacij. To so funkcije oblike 
\begin{equation}
\label{Mob_def}
    g(z) = \frac{az + b}{cz + d},
\end{equation}
kjer so $a$, $b$, $c$ in $d$ kompleksna števila, za katera velja $ad - bc \neq 0$.

Njihov pomen za opazimo, če si definiramo $s_1(z) = \frac{az + 1}{z} = a + \frac{1}{z}$ in $s_2(z) = \frac{bz + 1}{z} = b + \frac{1}{z}$; tedaj je namreč
\[
    s(z) = s_1(s_2(z)) = a + \cfrac{1}{b + \cfrac{1}{z}}  
\]
končen verižni ulomek in hkrati M\"{o}biusova transformacija. Transformacije bomo natančno definirali v poglavju \ref{sirse} in jih uporabili v dokazu glavnega izreka tega članka, za začetek pa si natančneje oglejmo verižne ulomke.

%%%%%%%%%%%%%%%%%%%%%%%%%%%%%%%%%%%%%%%%%%%%%%%%%%%%%%%%%%%%%%%%%%%%%%%%%%%%%%%%%

\section{Verižni ulomki}

V članku se bomo ukvarjali z enostavnimi verižni ulomki. Enostaven verižni ulomek ima vse $a_i = 1$, zanj vpeljemo tudi krajši zapis, ki je v končni obliki enak
\[
    [b_0, b_1, \ldots, b_n] = b_0 + \cfrac{1}{b_1 + \cfrac{1}{\cdots + \cfrac{1}{b_n}}},
\]

kjer je $b_0$ celo število, $b_1, b_2, \ldots$ pa naravna števila; enostaven verižni ulomek pa je nato enak

\begin{equation}
\label{Ver_def}
    [b_0, b_1, b_2, \ldots] = b_0 + \cfrac{1}{b_1 + \cfrac{1}{b_2 + \ddots}} = \lim_{n \to \infty} [b_0, b_1, \ldots, b_n].
\end{equation}

Limita v \eqref{Ver_def} vedno obstaja, saj njegovi približki $[b_0, b_1, \ldots, b_n]$ strogo naraščajo za sode $n$ in so navzgor omejeni s $[b_0, b_1]$ ter posledično konvergirajo. Za lihe $n$ pa ti približki strogo padajo in so omejeni z $[b_0]$ ter tako tudi konvergirajo. Ker je razlika med zaporednima približkoma obratno sorazmerna z $n^2$, sta limiti enaki, posledično pa limita \eqref{Ver_def} obstaja. 

Za verižni ulomek $[b_0, b_1, b_2, \ldots]$ rečemo, da je \emph{periodičen} s periodo $k$, če je $b_n = b_{n+k}$ za vsa naravna števila $n$, in da je \emph{sčasoma periodičen}, če je $b_n = b_{n+k}$ za vse dovolj velike $n$. Opazimo, da v primeru, ko je $[b_0, b_1, b_2, \ldots]$ periodičen s periodo $k$, kar označimo $\overline{[b_0, \ldots, b_{k-1}]}$ velja $b_0 = b_k \geq 1$. Člen $b_0$ je torej naravno število, vrednost periodičnega verižnega ulomka pa tako vedno večja od $1$. 

\subsection{Kvadratna iracionalna števila}

\begin{definicija}
	Realno število $x$ je \textbf{kvadratno iracionalno število} (kvadratni iracional), če je iracionalno število, ki je ničla kvadratnega polinoma $P$ s celoštevilskimi koeficienti.
\end{definicija}
Naj bo $x$ kvadratni iracional. Tedaj je ničla celoštevilskega kvadratnega polinoma in je zato oblike $x = \frac{a + b\sqrt{c}}{d}$, kjer so $a$, $b$, $c$ in $d$ cela števila, izmed katerih $b, c$ in $d$ ne smejo biti enaki nič, $c > 0$ pa ni popolni kvadrat. Ko vstavimo $x$ v kvadratni celoštevilski polinom $P$, za katerega je ničla, vidimo da je polinom do množenja s skalarjem enolično določen
Če je poljubno število take oblike ničla kvadratnega celoštevilskega polinoma vidimo, da je polinom do množenja s skalarjem enolično določen. Druga ničla polinoma $P$ je algebraična konjugirana vrednost $x$, t.j. $\frac{a - b\sqrt{c}}{d}$, ki jo označimo z $x^*$.

Za zapis realnih števil z enostavnimi verižnimi ulomki velja, da lahko vsako racionalno število zapišemo kot končen verižni ulomek, vsakemu iracionalnemu številu pa pripada enolično določen verižni ulomek oblike \eqref{Ver_def}. O verižnih ulomkih kvadratnih iracionalov lahko povemo še več, za iracionalno število $x = [b_0, b_1, b_2, \ldots]$ veljata naslednji lastnosti. \emph{Verižni ulomek $[b_0, b_1, b_2, \ldots]$ je sčasoma periodičen natanko tedaj, ko je $x$ kvadratni iracional,} in \emph{$[b_0, b_1, b_2, \ldots]$ je periodičen natanko tedaj, ko je x kvadratni iracional, katerega algebraična konjugirana vrednost $x^*$ leži na intervalu $(-1, 0)$.} Prvo ekvivalenco sta dokazala Euler, ki je pokazal, da sčasoma periodičen ulomek predstavlja kvadratni iracional, in Lagrange, ki je dokazal obrat. Drugo lastnost pa je pokazal Galois.

Osrednji namen tega članka je pokazati kako lahko s pomočjo M\"{o}biusovih transformacij na verižnih ulomkih dokažemo naslednji Galois-ev izrek.

\begin{izrek}[Galois-ev izrek]
	Za $x = \overline{[b_0, \ldots, b_{k-1}]}$ velja $\overline{[b_{k-1}, \ldots, b_0]} = - \frac{1}{x^*}.$
\end{izrek}

Oglejmo si primer uporabe transformacij, na posebnem primeru Galois-evega izreka.

\begin{zgled}
Naj $a, b \in \N$ in $\alpha = \overline{[a, b]}$. S substitucijo dobimo $\alpha = a + 1/(b + 1/\alpha).$ Torej je $\alpha$ negibna točka $s(z) = a + 1/(b + 1/z)$. Za negibni točki $s$ velja, da sta rešitvi enačbe $bz^2 - abz - a = 0$. To sta torej $\alpha$ in $\alpha^*$. Ker iz Vietovih formul sledi $\alpha\alpha^* = - \frac{a}{b} < 0$, velja $\alpha > 0 > \alpha^*$.  \\
Definirajmo še $\beta = \overline{[b, a]}$. Enak premislek nas pripelje do ugotovitve, da sta $\beta$ in $\beta^*$ rešitvi $az^2 - abz - b =0$ ter da velja $\beta > 0 > \beta^*$. Če na zadnji enačbi uporabimo transformacijo $w = - 1 / z$, dobimo $\{\alpha, \alpha^*\} = \{-1/\beta, -1/\beta^*\}$ in zato $\beta = -1/\alpha^*$, kar bi nam povedal tudi Galois-ev izrek.
\end{zgled}


%%%%%%%%%%%%%%%%%%%%%%%%%%%%%%%%%%%%%%%%%%%%%%%%%%%%%%%%%%%%%%%%%%%%%


\section{Širši pogled}
\label{sirse}

Pred dokazom izreka, si oglejmo delovanje in lastnosti kompleksnih funkcij, ravnine in v posebnem M\"{o}biusovih transformacij.

\subsection{Kompleksna ravnina}
Kompleksni ravnini $\C$ dodajmo novo točko $\infty$ in tako tvorimo \emph{razširjeno kompleksno ravnino,} ki jo označimo $\C_{\infty} = \C \cup \{\infty\}$.
\begin{definicija}
Funkcija $g$ z domeno $\C_{\infty}$ je \textbf{M\"{o}biusova transformacija}, če jo lahko zapišemo v obliki \eqref{Mob_def}, kjer so $a, b, c$ in $d$ kompleksna števila za katera velja $ad - bc \neq 0.$ \\
Če je $c \neq 0$, potem v \eqref{Mob_def} velja $g(\infty) = \frac{a}{c}$ in $g(-\frac{d}{c}) = \infty$, sicer je $g(\infty) = \infty.$
\end{definicija}

Vsaka M\"{o}biusova transformacija $g$ je bijekcija $\C_{\infty}$, saj je inverz funkcije oblike \eqref{Mob_def} enak $g^{-1}(z) = \frac{-dz + b}{cz - a}$. Vidimo da je $g^{-1}$ M\"{o}biusova transformacija, kratek račun pa nam utemelji, da to velja tudi za kompozitum dveh transformacij. Množica vseh M\"{o}biusovih transformacij je torej grupa.  

Označimo še \emph{razširjeno realno os} kot $\R_{\infty} = \R \cup \infty$. M\"{o}biusova transformacija ohranja $\R_\infty$ natanko tedaj, ko so vsi koeficienti v \eqref{Mob_def} realna števila. Iz predpostavke, da $g$ ohranja $\R_\infty$ sledi željeno zaradi ................................. PREMISLI!!

























\begin{definicija}
aaabbbbbb
\end{definicija}

\begin{zgled}
bbb
\begin{table}[h]
\[
\begin{array}{clc}
 n & \{1,2,\ldots, n\}          & \varphi(n)       \\
 \hline
 1 & \{{\bf 1}\}                    &     1      \\
 2 & \{{\bf 1},2 \}                &     1      \\
 3 & \{{\bf 1,2},3 \}             &     2      \\
 4 & \{{\bf 1},2,{\bf 3},4 \} &     2      \\
 5 & \{{\bf 1,2,3,4},5 \}       &     4      \\
 6 & \{{\bf 1},2,3,4,{\bf 5},6 \} &     2
\end{array}
\] 
\caption{Vrednosti funkcije $\varphi(n)$ za $n = 1,2,\ldots,6$}\label{fi}
\end{table}


\end{zgled}



\begin{trditev}
\label{fipp}
fipp
\end{trditev}

\noindent
{\em Dokaz:\/} ccc
\qed


\begin{zgled}
aaa
\end{zgled}


\begin{eqnarray*}
\varphi(n) &=& \prod_{i=1}^r \varphi\left(p_i^{k_i}\right)
\ =\ \prod_{i=1}^r \left(p_i^{k_i} - p_i^{k_i-1}\right) \\
 &=& \left(\prod_{i=1}^r p_i^{k_i}\right) \times \prod_{i=1}^r \left(1 - \frac{1}{p_i}\right)
\ =\ n \times \prod_{p\,|\,n} \left(1 - \frac{1}{p}\right). \qedm
\end{eqnarray*}


\begin{izrek}[Eulerjev izrek]
Euler
\end{izrek}

\noindent
{\em Dokaz:\/} ddd \qed


%%%%%%%%%%%%%%%%%%%%%%%%%%%%%%%%%%%%%%%%%%%%%%%%%%%%%%%%%%%%%%%%%%%%%


\begin{eqnarray*}
(f * (g + h))(n) &=& \sum_{d e = n} f(d)(g+h)(e)
\ =\ \sum_{d e = n} f(d)(g(e)+h(e)) \\
 &=& \sum_{d e = n} f(d)g(e) + \sum_{d e = n} f(d)h(e) \\
 &=& (f * g + f * h)(n). \qedm
\end{eqnarray*}


\section*{Angleško-slovenski slovar strokovnih izrazov}


\geslo{arithmetic function}{aritmetična funkcija}

\geslo{coprime}{tuj}

\geslo{Dirichlet convolution}{Dirichletova konvolucija}

\geslo{Dirichlet ring}{Dirichletov kolobar, kolobar aritmetičnih funkcij}

\geslo{divisor}{delitelj}

\geslo{Euler's phi function, Euler's totient function}{Eulerjeva funkcija $\varphi$}

\geslo{Euler's theorem}{Eulerjev izrek}

\geslo{Fermat's little theorem}{mali Fermatov izrek}

\geslo{fundamental theorem of arithmetic}{osnovni izrek aritmetike}

\geslo{greatest common divisor}{največji skupni delitelj, največja skupna mera}

\geslo{least common multiple}{najmanjši skupni večkratnik}

\geslo{M\"obius function}{M\"obiusova funkcija $\mu$}

\geslo{M\"obius inversion}{M\"obiusov obrat, M\"obiusova inverzija}

\geslo{multiple}{večkratnik}

\geslo{prime}{praštevilo; praštevilski}

\geslo{prime factor}{prafaktor}

\geslo{prime number}{praštevilo}

\geslo{relatively prime}{tuj}




\begin{thebibliography}{1}
\bibitem{AiZ}
M.~Aigner in G.~M.~Ziegler, \emph{Proofs from THE BOOK}, 2.\ izdaja, Springer, Berlin--Heidelberg--New York, 2001.
\bibitem{CaW}
N.~Calkin in H.~S.~Wilf, Recounting the rationals,
\emph{Amer.~Math.~Monthly}  \textbf{107}  (2000),  360--363.
\bibitem{Gra}
J.~Grasselli, \emph{Elementarna teorija števil}, DMFA -- založništvo, Ljubljana, 2009.
\end{thebibliography}





\end{document}