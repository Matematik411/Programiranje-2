%\documentclass[handout]{beamer}
\documentclass{beamer}

\usetheme{Madrid}
\usecolortheme{seahorse}
\useinnertheme{circles} 
\useoutertheme{split}
\setbeamertemplate{blocks}[rounded][shadow=true]

\usepackage[slovene]{babel}
\usepackage[OT2,T1]{fontenc}
\usepackage[utf8]{inputenc}
\usepackage{lmodern}
\usepackage{amsfonts,amsmath,amssymb,amsthm}
\usepackage{colortbl}
\usepackage[all]{xy}


\newtheorem{izrek}[theorem]{Izrek}
\newtheorem{trditev}[theorem]{Trditev}
\newtheorem{posledica}[theorem]{Posledica}
\newtheorem{vprasanje}[theorem]{Vpra\v anje}
\newtheorem{lema}[theorem]{Lema}
\newtheorem{definicija}{Definicija}

%\beamertemplatenavigationsymbolsempty

\title{M\"obiusove transformacije in periodični verižni ulomki}
\author[Nejc Zajc]{Nejc Zajc}
%\institute{University of Ljubljana, Slovenia}
\date{Seminar, 14. 4. 2020}

\begin{document}
%%%%%%%%%%%%%%%%%%%%%%%%%%%%%%%%%%%%%%%%
\begin{frame}
\maketitle

\end{frame}
%%%%%%%%%%%%%%%%%%%%%%%%%%%%%%%%%%%%%%%%

%%%%%%%%%%%%%%%%%%%%%%%%%%%%%%%%%%%%%%%%
\begin{frame}
\frametitle{Verižni ulomki}
Verižni ulomki so oblike
\[
    b_0 + \cfrac{a_1}{b_1 + \cfrac{a_2}{b_2 + \ddots}}.
\]  

\end{frame}
%%%%%%%%%%%%%%%%%%%%%%%%%%%%%%%%%%%%%%%%
\begin{frame}
\frametitle{Verižni ulomki}
Za celo število $b_0$ in naravna števila $b_1, b_2, \ldots$ je enostavni verižni ulomek 
\[
    [b_0, b_1, b_2, \ldots] = b_0 + \cfrac{1}{b_1 + \cfrac{1}{b_2 + \ddots}} = \lim_{n \to \infty} [b_0, b_1, \ldots, b_n].
\]
Njegove približke označujemo
\[
    c_n = [b_0, b_1, \ldots, b_n] = b_0 + \cfrac{1}{b_1 + \cfrac{1}{\cdots + \cfrac{1}{b_n}}}.
\]
\pause
Periodični verižni ulomek označimo $\overline{[b_0, \ldots, b_{k-1}]}$.





\end{frame}
%%%%%%%%%%%%%%%%%%%%%%%%%%%%%%%%%%%%%%%%
\begin{frame}
\frametitle{Kvadratna iracionalna števila}
\begin{definicija}
    Realno število $x$ je \textbf{kvadratno iracionalno število} (kvadratni iracional), če je iracionalno število in ničla kvadratnega polinoma $P$ s celoštevilskimi koeficienti.
\end{definicija}
\pause
\begin{trditev}
    Verižni ulomek $x = [b_0, b_1, b_2, \ldots]$ je sčasoma periodičen natanko tedaj, ko je $x$ kvadratni iracional.
\end{trditev}
\pause
\begin{trditev}
    Verižni ulomek $x = [b_0, b_1, b_2, \ldots]$ je periodičen natanko tedaj, ko je $x$ kvadratni iracional, katerega algebraična konjugirana vrednost $x^*$ leži na intervalu $(-1, 0)$.
\end{trditev} 




\end{frame}

%%%%%%%%%%%%%%%%%%%%%%%%%%%%%%%%%%%%%%%%
\begin{frame}
\frametitle{Galois-ev izrek}
\begin{izrek}[Galois-ev izrek]
    \label{Galois}
    Za $x = \overline{[b_0, \ldots, b_{k-1}]}$ velja $\overline{[b_{k-1}, \ldots, b_0]} = - \frac{1}{x^*}.$
\end{izrek}
\end{frame}



%%%%%%%%%%%%%%%%%%%%%%%%%%%%%%%%%%%%%%%%
\begin{frame}
\frametitle{M\"obiusova transformacija}
\begin{definicija}
    Funkcija $g$ z domeno $\mathbb{C}_{\infty} = \mathbb{C} \cup \{\infty\}$ je \textbf{M\"obiusova transformacija}, če jo lahko zapišemo v obliki
    \[
        g(z) = \frac{az + b}{cz + d},
    \]
    kjer so $a$, $b$, $c$ in $d$ kompleksna števila za katera velja $ad - bc \neq 0$. \\
    Če je $c \neq 0$, potem velja $g(\infty) = \frac{a}{c}$ in $g(-\frac{d}{c}) = \infty$, sicer je $g(\infty) = \infty$.
\end{definicija}
\end{frame}

%%%%%%%%%%%%%%%%%%%%%%%%%%%%%%%%%%%%%%%%
\begin{frame}
\frametitle{Modularna grupa}
\begin{definicija}
    \textbf{Modularna grupa} $\Gamma$ je grupa vseh M\"obiusovih transformacij s celoštevilskimi koeficienti $a$, $b$, $c$ in $d$, za katere velja $ad - bc = 1$.
\end{definicija}
\pause
\begin{definicija}
    Loksodromične izometrije $\mathbb{H}$ so M\"obiusove transformacije, ki ohranjajo $\mathbb{H}$ in imajo dve različni negibni točki. 
\end{definicija}
\pause
\begin{trditev}
    Realno število $x$ je kvadratno iracionalno število natanko tedaj ko je negibna točka nekega loksodromičnega elementa $g$ modularne grupe $\Gamma$.\\
    Tedaj je algebraična konjugirana vrednost $x^*$ druga negibna točka $g$.
\end{trditev}

\end{frame}

%%%%%%%%%%%%%%%%%%%%%%%%%%%%%%%%%%%%%%%%
\begin{frame}
\frametitle{Dokaz izreka}
\begin{lema}
    Za funkcije $s_i$, $i \in \{1, \ldots, k\}$ oblike $s : z \mapsto b + 1/z$, kjer je $b \geq 1$, ima končni kompozitum $S = s_1 \cdots s_k$ privlačno negibno točko $\zeta \in (1, \infty)$ in odbojno negibno točko $\tilde{\zeta} \in (-1, 0)$. 
\end{lema}

\end{frame}
%%%%%%%%%%%%%%%%%%%%%%%%%%%%%%%%%%%%%%%%
\begin{frame}
\frametitle{Dokaz izreka}
\begin{izrek}[Galois-ev izrek]
    Za $x = \overline{[b_0, \ldots, b_{k-1}]}$ velja $\overline{[b_{k-1}, \ldots, b_0]} = - \frac{1}{x^*}.$
\end{izrek}

\end{frame}
%%%%%%%%%%%%%%%%%%%%%%%%%%%%%%%%%%%%%%%%
% končen slide?

%%%%%%%%%%%%%%%%%%%%%%%%%%%%%%%%%%%%%%%%
\end{document}
